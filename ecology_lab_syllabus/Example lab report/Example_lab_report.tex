\documentclass[]{article}
\usepackage{lmodern}
\usepackage{amssymb,amsmath}
\usepackage{ifxetex,ifluatex}
\usepackage{fixltx2e} % provides \textsubscript
\ifnum 0\ifxetex 1\fi\ifluatex 1\fi=0 % if pdftex
  \usepackage[T1]{fontenc}
  \usepackage[utf8]{inputenc}
\else % if luatex or xelatex
  \ifxetex
    \usepackage{mathspec}
  \else
    \usepackage{fontspec}
  \fi
  \defaultfontfeatures{Ligatures=TeX,Scale=MatchLowercase}
\fi
% use upquote if available, for straight quotes in verbatim environments
\IfFileExists{upquote.sty}{\usepackage{upquote}}{}
% use microtype if available
\IfFileExists{microtype.sty}{%
\usepackage{microtype}
\UseMicrotypeSet[protrusion]{basicmath} % disable protrusion for tt fonts
}{}
\usepackage[margin=1in]{geometry}
\usepackage{hyperref}
\hypersetup{unicode=true,
            pdftitle={Example lab report},
            pdfauthor={Stephen Cook},
            pdfborder={0 0 0},
            breaklinks=true}
\urlstyle{same}  % don't use monospace font for urls
\usepackage{graphicx,grffile}
\makeatletter
\def\maxwidth{\ifdim\Gin@nat@width>\linewidth\linewidth\else\Gin@nat@width\fi}
\def\maxheight{\ifdim\Gin@nat@height>\textheight\textheight\else\Gin@nat@height\fi}
\makeatother
% Scale images if necessary, so that they will not overflow the page
% margins by default, and it is still possible to overwrite the defaults
% using explicit options in \includegraphics[width, height, ...]{}
\setkeys{Gin}{width=\maxwidth,height=\maxheight,keepaspectratio}
\IfFileExists{parskip.sty}{%
\usepackage{parskip}
}{% else
\setlength{\parindent}{0pt}
\setlength{\parskip}{6pt plus 2pt minus 1pt}
}
\setlength{\emergencystretch}{3em}  % prevent overfull lines
\providecommand{\tightlist}{%
  \setlength{\itemsep}{0pt}\setlength{\parskip}{0pt}}
\setcounter{secnumdepth}{0}
% Redefines (sub)paragraphs to behave more like sections
\ifx\paragraph\undefined\else
\let\oldparagraph\paragraph
\renewcommand{\paragraph}[1]{\oldparagraph{#1}\mbox{}}
\fi
\ifx\subparagraph\undefined\else
\let\oldsubparagraph\subparagraph
\renewcommand{\subparagraph}[1]{\oldsubparagraph{#1}\mbox{}}
\fi

%%% Use protect on footnotes to avoid problems with footnotes in titles
\let\rmarkdownfootnote\footnote%
\def\footnote{\protect\rmarkdownfootnote}

%%% Change title format to be more compact
\usepackage{titling}

% Create subtitle command for use in maketitle
\newcommand{\subtitle}[1]{
  \posttitle{
    \begin{center}\large#1\end{center}
    }
}

\setlength{\droptitle}{-2em}

  \title{Example lab report}
    \pretitle{\vspace{\droptitle}\centering\huge}
  \posttitle{\par}
    \author{Stephen Cook}
    \preauthor{\centering\large\emph}
  \postauthor{\par}
      \predate{\centering\large\emph}
  \postdate{\par}
    \date{January 9, 2019}


\begin{document}
\maketitle

\hypertarget{introduction}{%
\section{Introduction}\label{introduction}}

Student success in college is more critical today than at any point in
history. Postsecondary education is increasingly necessary to become
economically self-sufficient, and the baccalaureate degree is the
preferred choice for many. However, many students are not as prepared
academically as collegiate faculty and teaching-assistants (TAs) would
like. The syllabus is a common tool used in virtually every college
course to outline expectations and ways that the course or lab is
designed to help their academic development.

The objective of this experiment was to determine the effectiveness of
the syllabus in aiding student success in Biology 3103 (BIO 3103). We
tested the null hypothesis that students who receive the syllabus, and
those that do not, receive similar scores on their Lab Reports. Our
alternative hypothesis was that students who read the syllabus would
perform significantly better that students who did not.

\hypertarget{methods}{%
\section{Methods}\label{methods}}

We conducted a study on two sections of BIO 3103 in the spring of 2019
using Lab Report scores as a measure of student success in the course.
To determine the effectiveness of the syllabus, we distributed the
syllabus to one section, and withheld the syllabus from the other
section. The syllabus contained information about lab meetings,
instructor contact information, scheduling, report expectations, and
information about how to submit work electronically. There were 14
students in the section receiving the syllabus, and 18 students in the
section not receiving the syllabus. Each student completed 6 reports
throughout the semester. All other variables such as the number of lab
meetings, field excursions, and lab report grading were held constant
between the two sections throughout the semester. At the end of the
semester, lab report scores (as a point score ranging from 0-12) were
compared between the two sections.

We calculated the mean and standard error (s.e.) of the report scores
for each section, and displayed this information in a bar-graph. To
determine if there was any statistically significant difference between
the section scores, we conducted a two-sample t-test (\emph{a priori}
\$\alpha of 0.5).

\hypertarget{results}{%
\section{Results}\label{results}}

The syllabus had a significant effect on lab report scores of students
in BIO 3103 (t-test, p\textless{}0.001). The mean lab report score of
students who had access to the lab syllabus (10.15 +/- 0.17 points) was
higher than students who did not (8.23 +/- 0.18 points).

\begin{figure}
\centering
\includegraphics{Example_lab_report_files/figure-latex/unnamed-chunk-7-1.pdf}
\caption{Figure 1. Access to the course syllabus had a significant
effect on lab report scores (t-test, p\textless{}0.001). Shown are group
means +/- the standard error. Mean lab report scores were higher for
students with syllabus access (10.15 +/- 0.17 points) than students with
no access (8.26 +/- 0.18 points)}
\end{figure}

\hypertarget{discussion}{%
\section{Discussion}\label{discussion}}

We rejected the null hypothesis that the lab report syllabus has no
effect on student success. The data supported our alternative
hypothesis, that students who read the syllabus earn significantly
higher scores on their lab reports. That students earn higher scores on
lab reports after reading the syllabus is probably because of the
information the syllabus contains about what is required in lab reports.
Students who were denied access to the syllabus frequently left out
important information required in every lab report such as hypothesis
statements and detailed figure captions.

The variability in scores between the two sections was similar (0.17
s.e. for the syllabus group, and 0.18 s.e. for the non syllabus group).
This could mean that though students who received the syllabus scored
higher, some students do particularly well or poor despite access to the
syllabus. Additional factors could have influenced final scores on lab
reports such as class history, the number of class hours that a student
was enrolled in, and the number of classes the student missed. Syllabi
are important tools used in education, and our results indicate that
they should be distributed in every class to enhance student success.


\end{document}
